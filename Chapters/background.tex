\section{Background}
The use of drugs has changed throughout the years, and many new substances have been introduced. With the growth of technology, the Internet is the preferred platform used by drug traffickers. According to the \cite{medData} the number of people taking multiple drugs simultaneously has increased, which is a big concern. The increasing trend of people taking multiple drugs simultaneously not only raises the risk of more mental and physical health problems but also increases the chances of experiencing multiple health issues and increases the risk of premature death. \cite{medData} highlight that one major problem that the United States is facing is the number of people who dies from the use of opioid. In the US one every eleven minutes dies due to an overdose. 

% Which country is the highest?
\cite{medData} states that in Sweden there are multiple warnings from several quarters regarding the use of a new “gateway drug”. The new drug is a combination of tramadol (an opioid) and pregabalin (Lyrica) and is used by both young people and the elderly. One major concern is the high drug-related mortality rate in Sweden, according to \cite{medData}, there is only one country in Europe that has a higher drug-related mortality rate than Sweden. The concern has been noticed by several international authorities and bodies as well as by the government, various authorities, media, clinics, and researchers nationally. Sweden reported 900 people died per year from the years 2012-2018. The average number of deaths in Sweden was almost three people per year, and the cause was drug poisoning. This is a major concern and therefore it is important to continuously updated knowledge in the field.

In 2019, the National Board of Health and Welfare was informed and tasked by the government to find a countering approach to decrease the number of drug-related deaths. In March 2020, a report on drug/narcotic-related mortality was published by the National Board of Health and Welfare. The report accordingly to \cite{medData} contained data from the authority's own register data. In research where data from different registers are collected and collated, there is the possibility of generating more exhaustive knowledge about changes in drug-related mortality, for example for different geographical areas, subgroups within the population, etcetera. In this way, \cite{medData} states that patterns can be discerned that show changes over time regarding which drugs/mixtures were ingested/were most frequently occurring in deaths, registered cause of death as well as information on age, sex, geographic region, socio-economic factors, previous treatment interventions, crime, and more.

Given the demand for knowledge in this field, the project aims to contribute significantly to understanding drug-related mortality in Sweden. The expected results have important clinical implications and provide valuable insights for problem solving and assisting professionals in social services, law enforcement, and healthcare.
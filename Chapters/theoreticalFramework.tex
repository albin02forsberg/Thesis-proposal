\section{Theoretical Framework}

The study primarily focuses on the dynamic of psychotropic drug use and its impact on the Swedish mortality rate and therefore aims to address the following concerns: the emergence of new substances, the increase in internet-based drug trafficking, and the possible health risks that come with the use of multiple drugs. \cite{medData} states that the urgency of studying drug-related mortality comes from the opioid crisis in the United States. When it comes to Sweden, \cite{medData} highlights two main concerns: the country's high drug-related mortality rates and the use of tramadol and pregabalin as potential "gateway drugs". \cite{medData} also states that the National Board of Health and Welfare has been contacted by the Swedish government to provide information that can help them prevent and take care of health issues. 


The study will be performed on a set of medical data from 2012 to 2020 and will examine the changes in drug-related mortality from that period. It will also explore different patterns across both socio-demographic groups and geographical regions. The hypothesis for the study will explore and reflect on consumption patterns and their effects on mortality, and if the potential efforts to prevent and change in rules can make an impact on the subject. As mentioned earlier in the text, the study will investigate changes in population characteristics, associations with preventive measures, and variations across geographical areas and subpopulations. 

The following variables were provided by \cite{medData} to use in the study: gender, age at death, substances found at autopsy, prescribed medications, psychiatric diagnoses, external causes of death, method and place of death, and socio-economic variables. There are also Time-related considerations such as including the year of autopsy and the year before. % What does you mean by the year before?

Through the research, the authors seek to help law enforcement, social services, and healthcare by providing crucial insights and information on the subject. 

\section{Background}
With the continuous growth of technology, more possibilities are born with the help of algorithms. The algorithms are instructions for the AI to carry out different tasks which we then use as a part of our everyday life. This is very positive for the evolution of AI, one of the most common chatbots is ChatGPT which can be used to answer questions and write text on different subjects. Another example is ChatPDF, this chatbot takes in a PDF file, and with instruction from the user, it can summarize and answer questions. These are just a few examples of how algorithms can be used to make our tasks easier. Are there any available chatbots for analyzing data for business?

When it comes to analyzing data for business, it needs to be divided into two parts: Storing data and then analyzing the stored data. Let us start with storing data, one of the ways major companies use to store data is by Cloud. By using the Cloud, you can store and access data easily through the Cloud.  The formula behind Cloud is that it gives you a certain amount of GB (Gigabytes) of memory to store data for free. Afterward, for each GB of memory wanted by a company, a certain amount of money needs to be paid. This makes it difficult for small businesses that do not have access to large resources. If hypothetically the first part is established, then the data needs to be analyzed and provide different statistics for the user. The current tools available on the internet are few and are free to a certain degree. To access certain features a small fee needs to be paid.  What problems are businesses facing?

One thing to take into consideration is that all businesses do not have access to large amounts of funds to spend on either Cloud storage to store data or software to display statistics. One possible solution is to build a free-to-use website where the user stores data in the database. Algorithms will be created to use the stored data and create different types of charts to visualize statistics. Is this achievable?

The only thing stopping us from creating algorithms is the amount of time given, if we start implementing our idea in January then it is possible otherwise another approach is needed. Instead of algorithms, a website will be created with the data manually presented through different charts.  

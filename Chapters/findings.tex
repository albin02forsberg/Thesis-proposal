\section{Expected Findings}
The research aims to provide important information about drug-related addictions by analyzing patterns found in the data set. Additionally, it seeks to contribute to the broader fields of data visualization and big data analytics. 

\subsection{Expected Findings
}
Analysis of the drug addiction data set from 2012 to 2020 is expected to reveal nuances in demographic characteristics, associations with prevention efforts, and differences among geographic regions and subpopulations. Expected results include insights into factors influencing changes in drug use trends, the effectiveness of prevention interventions, and regional or subgroup preferences for drug-related substances.


\subsection{Contribution to the Field}
The contribution from the research aims to impact and address the pressing concerns outlined by \cite{medData}, namely the mortality rate and the use of “gateway drugs”. By integrating data methodology visualization and big data analysis, we aim to create a free-to-use website that will include a full analysis of the medical data which hopefully can be used as a tool for physicians and researchers alike.

\subsection{Relevance to the Main Field of Study}
With \cite{medData} highlighting the high drug-related mortality rate in Sweden, there is an urgent need for monitoring and analyzing the data set given to us. By combining informatics expertise with healthcare insight, the research aims to provide local hospitals with a possible practical solution and to provide valuable insights and recommendations obtained from their medical data set. The outcome of this research is aimed to be used as the basis for policy adjustments, prevention strategies, and health measures and hopefully contribute to the general well-being of the Swedish people who suffer from drug-related addictions. 

In summary, the expected result from the study will not only address the concern highlighted by \cite{medData}, but it will also advance the development of data visualization methodologies and make a meaningful contribution to a major research area. 


	
